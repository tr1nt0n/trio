\documentclass[10pt]{article}
\usepackage{fontspec}
\usepackage[utf8]{inputenc}
\setmainfont{Bodoni 72 Book}
\usepackage[paperwidth=9in,paperheight=12in,margin=1in,headheight=0.0in,footskip=0.5in,includehead,includefoot,portrait]{geometry}
\usepackage[absolute]{textpos}
\TPGrid[0.5in, 0.25in]{23}{24}
\parindent=0pt
\parskip=12pt
\usepackage{nopageno}
\usepackage{graphicx}
\graphicspath{ {./images/} }
\usepackage{amsmath}
\usepackage{tikz}
\newcommand*\circled[1]{\tikz[baseline=(char.base)]{
            \node[shape=circle,draw,inner sep=1pt] (char) {#1};}}

\begin{document}

\begingroup
\begin{center}
\huge NOTES FOR THE INTERPRETERS
\end{center}
\endgroup

\begingroup
\begin{center}
\huge 
\end{center}
\endgroup


\begingroup
\textbf{General: 1.)} Dynamics in this score are effort dynamics, representing the physical force behind an action rather than the sounding dynamic. \textbf{2.)} Stem tremoli are to be performed as quickly as possible, and do not represent a subdivision of a note. \textbf{3.)} Dashed arrows above the staff indicate a gradual transition from one technique to another. \textbf{4.)}  Playing techniques persist until cancelled by another technique, with the exception of the sul IV direction in the cello at measure 37, which is terminated in the following section after the double bar line. \textbf{5.)}  All instruments read two staves, with the top stave indicating the actions of the right hand and the bottom stave indicating the actions of the left hand. If one of the staves is empty, the division of the hands is at the discretion of the interpreter. \textbf{6.)} Stemless note heads enclosed in a duration bracket are to be performed in free rhythm, so long as all the notes are played within the duration indicated by the bracket. \textbf{7.)} Time signatures whose denominators are not a power of two are to be understood as a type of metric modulation wherein the pulse shifts to a prolation indicated by the denominator. For example, \textbf{1/6} will contain one ``sixth" note, which is one-sixth of a whole note, or, a triplet quarter note. When these time signatures are active, tuplet brackets which are open on the right side similarly indicate the prolation of a note alone, rather than the number of beats in the prolation. \textbf{8.)} Sections delineated by double bar lines and rehearsal marks are to be understood as separate movements, but should be played attacca, especially maintaining the dynamic transitions between the movements.\\\
\endgroup

\begingroup
\textbf{Strings: 1.)} The actions of the right hand are indicated by a four-line stave, wherein the top line indicates to play on string I, the next indicates to play on string II, and so on. \textbf{2.)} The bow speed indications in this score are \textbf{extra fast bow, or XFB,} which indicates almost an irregular tremolo, moving the bow as quickly and with as full strokes as possible,  \textbf{fast bow, or FB,} which indicates to bow at flautando speed, though not necessarily sul tasto, \textbf{normale bow, or NB,}  which indicates normale bow speed, and \textbf{extra slow bow, or XSB,} which indicates to bow as slowly as possible, generating scratch tone at higher bow pressures. \textbf{3.)}  The horizontal angle of the bow is indicated by degree articulations, wherein -45° indicates pointing the tip of the bow as far downward as is comfortable, and 45° indicates pointing the tip as far upward as is comfortable. This notation is used most in the movement, ``in th posession of nymphs and naiads,'' to create a tasto/ponticello contrast between the strings being alternated. In the absence of these indications, bow angle is left to the discretion of the interpreter. \textbf{4.)} When performing the ``Pull" direction, interpreters should hook their finger around the string at the pitch indicated, and pull the string upwards until the pitch is bent to the pitch of the following note, after which releasing the string to create a snap pizzicato. \textbf{5.)} Flat glissandi are sometimes used for the same function as ties. \textbf{6.)} Dashed slurs indicate to play a passage legato without indicating a particular bowing. \textbf{7.)} Spectral microtones are indicated by a cent-deviation articulation printed above an equally tempered note. In the absence of electric tuners, approximations of these deviations are acceptable. \textbf{8.)} Finger pressure of the left hand is indicated by note head shape, wherein traditional note heads indicate a fully closed string, triangle-shaped note heads indicate a pressure half-way between harmonic pressure and fully closing the string, and diamond-shaped note heads indicate to touch the notated pitch with pressure as if playing a harmonic, whether a harmonic sounds or not. \textbf{9.) Molto sul ponticello, or MSP} indicates to play with half of the bow hair directly on the bridge and half of the hair on the string, \textbf{sul ponticello, or SP} indicates to bow near the bridge, \textbf{sul tasto, or ST} indicates to bow above the edge of the fingerboard, \textbf{molto sul tasto, or MST} indicates to bow as close to the fingers as possible, \textbf{1/2 col legno tratto, or 1/2 CLT} indicates to bow with both the hair and the wood touching the string, and \textbf{Crine} cancels 1/2 col legno tratto. \textbf{10.)} Circular bowing is indicated by a circled arrow articulation, such as the articulation at measure 71. \\
\endgroup

\begingroup
\textbf{Piano: 1.)} Liberal use of the sustain pedal is encouraged, so long as rests remain silent. \textbf{2.)} Arpeggi are not to be performed in any given direction or order. Instead, the interpreter should choose a sporadic order which should differ from one chord to the next. When combined with a stem tremolo this should create an unpredictable rumbling or bubbling affect. \\
\endgroup

\end{document}